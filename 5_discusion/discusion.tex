\section{Discusión.}\label{sec:5_discusion}

%En la discusión se resumen, interpretan y extrapolan los resultados, se analizan sus implicaciones y limitaciones, y se confrontan con las hipótesis planteadas, considerando cómo ha sido la perspectiva de otros autores.

%En otras palabras, se hace énfasis en aspectos resumidos y escuetos del estudio, planteamiento de propuestas de investigaciones futuras, comparación con otros estudios, presentación de las limitaciones del estudio y de la posible generalización de los resultados, de otros hallazgos no previstos y de la interpretación de los resultados por el investigador, entre otros aspectos. Este artículo se centrará en desarrollar específicamente este tópico.

Los sistemas de inyección de jeringa han mostrado ser un método más práctico para nuestro propósito que los sistemas de control de presión. Los flujos que nosotros hemos encontrado como óptimos en la generación de las gotas son parecidos a los que hacen referencia otros autores como \cite{article:hirose}, que utilizan un chip con una simetría y tamaño de canal al chip de la Figura~\ref{subfig:chip3}. 

Se ha mostrado que somos capaces de obtener una muestra de gotas de agarosa con una concentración del 1\% con un diámetro muy próximo a las 125 micras y una desviación estándar que ronda el 5\% (ver Sección~\ref{muestra10x}). Los autores \cite{article:hirose} aseguran haber conseguido gotas de agrosa a una concentración del 3\% y un diámetro de 50~micras con una desviación estándar entorno al 1\%. En el caso de \cite{article:haakan} citan a autores que también consiguen desviaciones en el tamaño de entre el 1\% y el 3\%, pero no nos informan sobre la composición de la fase dispersa; en \cite{article:ralf} se habla de desviaciones de entrono al 2\% pero de nuevo, tampoco nos indican la composición de la fase dispersa. Es necesario un estudio minucioso de la metodología para identificar qué podemos mejorar para conseguir una mejor monodispersidad de las gotas. 

En numerosos artículos señalan que la distribución de encapsulados debe seguir una distribución de Poisson, partiendo de una muestra con las características descritas en la Sección~\ref{subsec:concentracion_celulas}. En el artículo de \cite{article:haakan} nos indica que el promedio células por gota (el valor de $\lambda$ en la distribución de Poisson) debe situarse entre 0.1 y el 1. En \cite{article:sarah} nos indican que, dependiendo de la aplicación, lo que se busca típicamente es una distribución con $\lambda=0.1$, $\lambda=0.3$ o $\lambda=0.5$. Nosotros somos capaces de encapsular dentro de ese rango de valores de $\lambda$.
El \anglicismo{script} que se presenta en este trabajo, relaciona directamente la densidad de células con el valor $\lambda$. Una vez conocemos el valor de $\lambda$ que es interesante para nuestros experimentos, se puede hacer un ajuste como el que se muestra en la Figura~\ref{fig:rho_d}, lo cual proporciona un método rápido, sencillo y preciso para calcular la densidad de células que necesitamos en la muestra a encapsular.
En la muestra~1x lo que estábamos buscando es un valor de $\lambda=0.098$ pero debido a que el tamaño de las gotas fue diferente a lo que habíamos supuesto antes de hacer el experimento y debido a que quizás la densidad de células no era del todo correcta, se consiguió un $\lambda=0.129$. En el caso de la muestra~10x, la concentración era 10 veces superior, por lo que buscábamos un valor de $\lambda=0.98$ y el valor obtenido fue $\lambda=0.966$, como vemos un valor muy próximo. Esto quiere decir que el método para calcular las densidades celulares de muestra a fin de conseguir una frecuencia de encapsulado óptima resulta ser adecuado. También prueba que lo que se ha identificado en los recuentos con la cámara de Nueubauer como células, es lo que se ha identificado como célula en los recuentos realizados con el microscopio de epifluorescencia.

Dependiendo de la finalidad con la que realicemos los encapsulados, es posible que tengamos que ajustar muchos de los parámetros que se han utilizado en este trabajo. Sin embargo, los encapsulados que hemos conseguido tienen unas características compatibles con los que son aceptadamente válidos para realizar análisis genómico de células individuales.


% Conteo de las células con la cámara (alternativas mas precisas)

% Conteo de las células en las imágenes. Problema de los restos celulares en la muestra y la dificultad de (automatizar el proceso.)



%Sistema de presienoes frente a las jeringas.

% Tengo que discutir sobre los objetivos y las características de la plataforma en si y los resultados obtenidos.