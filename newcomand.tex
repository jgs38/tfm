%    NUEVOS COMANDOS DEFINIDOS MEDIANTE NEWCOMAND
\DeclareMathOperator\erf{erf}

% ENTORNO MATEMÁTICO
\newcommand{\maths}[1]{$ #1 $}

% TEXTO RESALTADO
\newcommand{\resalta}[1]{\textit{#1}}
\newcommand{\anglicismo}[1]{\textit{#1}}
\newcommand{\cultismo}[1]{\textit{#1}}
\newcommand{\comillas}[1]{``#1''}

% SÍMBOLOS MATEMÁTICOS
\newcommand{\menoroigual}{$\geqslant$}
\newcommand{\delorden}{$\sim$}
\newcommand\norm[1]{\left\lVert#1\right\rVert}
\newcommand{\porcentaje}[1]{$#1\:\%$}

%\lesssim  	less or quivalent
% \gtrsim 

% UNIDADES
%   PROPIAS DE ASTRONOMÍA
\newcommand{\solares}[2]{$#2 \:{#1}_{\odot}$}
\newcommand{\masassolares}[1]{$#1 \:{M}_{\odot}$}
\newcommand{\luminosidadessolares}[1]{$#1 \:{L}_{\odot}$}
\newcommand{\tfe}[1]{$#1 \:{M}_{\odot}\,\mathrm{{yr}^{-1}}$}
% \newcommand{\myr}{}
\newcommand{\flujo}[1]{${S}_{#1 \:\mu\mathrm{m}}$}
%   GENERALES
\newcommand{\microm}[1]{$#1 \:\mu\mathrm{m}$}
\newcommand{\angstrom}[1]{$#1 \:$\AA}
\newcommand{\mjy}[1]{$#1 \:\mathrm{mJy}$}
\newcommand{\jy}[1]{$#1 \:\mathrm{Jy}$}
\newcommand{\kelvin}[1]{$#1 \:\mathrm{K}$}
%   Grados celsius
\newcommand{\grad}{$^{\circ}$}

% NOMBRES USADOS FRECUENTEMENTE
\newcommand{\hatlas}{\mbox{\textit{H}-ATLAS}}
\newcommand{\halos}{\mbox{HALOS}}
\newcommand{\halo}{\mbox{HALO}}
\newcommand{\gama}{\mbox{GAMA}}
\newcommand{\spire}{\mbox{SPIRE}}
\newcommand{\pacs}{\mbox{PACS}}
\newcommand{\hifi}{\mbox{HIFI}}
\newcommand{\h}{\textit{\mbox{Herschel}}}
\newcommand{\rt}{\textit{\mbox{redshift}}}
\newcommand{\rts}{\textit{\mbox{redshifts}}}
\newcommand{\sed}{\mbox{SED}}
\newcommand{\sfr}{\mbox{SFR}}
\newcommand{\slg}{\mbox{SLG}}
\newcommand{\etg}{\mbox{ETG}}
\newcommand{\cross}{\mbox{cross-identificación}}
\newcommand{\smm}{\mbox{SMM~J2135-0102}}
\newcommand{\arp}{\mbox{Arp220}}
\newcommand{\gquince}{\mbox{G15.141}}

\newcommand{\typewriter}[1]{\texttt{#1}}
\newcommand{\python}{\texttt{\mbox{Python}}}
\newcommand{\script}{\textit{\mbox{script}}}

% PARÁMETROS
\newcommand{\z}{\textit{z}}
\newcommand{\paramk}{'K'}
\newcommand{\paramc}{'C'}

%   Compuestos químicos
\newcommand{\agua}{H_{2}O}

%   Nuevos comandos para tablas
%http://tex.stackexchange.com/questions/12703/how-to-create-fixed-width-table-columns-with-text-raggedright-centered-raggedlef
\newcolumntype{L}[1]{>{\raggedright\let\newline\\\arraybackslash\hspace{0pt}}m{#1}}
\newcolumntype{C}[1]{>{\centering\let\newline\\\arraybackslash\hspace{0pt}}m{#1}}
\newcolumntype{R}[1]{>{\raggedleft\let\newline\\\arraybackslash\hspace{0pt}}m{#1}}

%   Modificación de la linea para el pie de página
%https://es.wikibooks.org/wiki/Manual_de_LaTeX/Gestionando_la_bibliograf%C3%ADa/Notas_al_pie
\renewcommand{\footnoterule}{\vspace*{-3pt}
  \noindent\rule{5cm}{1pt}\vspace*{2.6pt}}
  
% densidad del cultivo
\newcommand{\rhocultivo}{\mbox{${\rho}_c$}}
% densidad de la muestra
\newcommand{\rhomuestra}{\mbox{${\rho}_c$}}

% celulas/mm3
\newcommand{\celpormmcubico}[1]{$cel {mm}^{-3}$}
% celulas/ml
\newcommand{\celporml}{\hspace{-0,5mm}$\mathrm{cel {ml}^{-1}}$}
% micrometros
\newcommand{\micrometro}{\hspace{-0,5mm}$\mathrm{\mu m}$}
% microlitros
\newcommand{\microlitro}{\hspace{-0,5mm}$ \mathrm{\mu l}$}
% picolitros
\newcommand{\picolitro}{\hspace{-0,5mm}$\mathrm{pl}$}
% litro
\newcommand{\litro}{\hspace{-0,5mm}$\mathrm{l}$}
% Grados Celsius
\newcommand{\celsius}{\hspace{-2,5mm}$\phantom{a}^{\circ} \mathrm{C}$}
% Omios
\newcommand{\ohm}{\hspace{-0,5mm}$\Omega$}
% KiloOmios
\newcommand{\kiloohm}{\hspace{-0,5mm}$ \mathrm{k\Omega}$}
% kilovoltios/mm
\newcommand{\kilovoltiospormm}{\hspace{-0,5mm}$\mathrm{kW}$}
\newcommand{\microlitrosporminuto}{\hspace{-0,5mm}$\mathrm{\mu l\,{min}^{-1}}$}
% nanometros
\newcommand{\nanometro}{\hspace{-0,5mm}$\mathrm{nm}$}
% milimetros
\newcommand{\milimetro}{\hspace{-0,5mm}$\mathrm{mm}$}
% milimetros
\newcommand{\milimetrocubico}{\hspace{-0,5mm}$\mathrm{{mm}^{3}}$}
% watios
\newcommand{\vatio}{\hspace{-0,5mm}$ \mathrm{W}$}
% gotas, gotitas, microgotas, droplets
\newcommand{\gotas}{\mbox{microgotas}}

% gota, gotita, microgota, droplet
\newcommand{\gota}{\mbox{microgota}}

  
  
  
  