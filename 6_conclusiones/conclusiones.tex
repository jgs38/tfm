\section{Conclusiones.}\label{sec:6_conclusiones}

Como conclusiones más destacadas de este trabajo cabe indicar las siguientes:

\begin{enumerate}
    \item Hemos conseguido poner a punto un sistema de encapsulación de células individuales en microgotas de aceite. Este sistema permite la incorporación de estas células en esferas de agarosa.
    
    \item Dicho sistema muestra un funcionamiento óptimo con inyectores de flujo controlado (microinyectores) frente a sistemas de control por presión.
    
    \item Hemos puesto a punto un sistema para el control de la temperatura de la agarosa, lo que nos permite obtener flujos estables en el proceso de encapsulación.
    
    \item Hemos desarrollado un simulador de las condiciones óptimas de encapsulación. La validación experimental demostró que con unos flujos de 6~\microlitrosporminuto\ para la fase continua y 2.25~\microlitrosporminuto\ para la fase dispersa es posible obtener gotas de un diámetro de 124 micras, con una desviación estándar de 6 micras.
    
    \item La concentración que nosotros hemos considerado óptima para encapsular con gotas de 125 micras de diámetro es de $\sim98\cdot{10}^{3}\;\mathrm{cel\;mm^{-3}}$ que se corresponde con $\lambda=0.098$. 
    
\end{enumerate}

\nocite{article:freeman}
\nocite{article:ralf}
\nocite{article:pingan}
\nocite{article:huifa}
\nocite{article:ye-jin}
\nocite{article:sarah}
\nocite{article:haihu}
\nocite{book:andreas}
\nocite{article:piotr}
\nocite{article:andrew_s}
\nocite{article:richard}
\nocite{article:hirose}
\nocite{article:haakan}