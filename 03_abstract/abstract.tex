\selectlanguage{spanish}

\begin{abstract}
Las tecnologías de secuenciación de próxima generación (NGS) permiten la determinación del genoma completo o del transcriptoma presente en una muestra biológica. Sin embargo, muchas muestras biológicas son heterogéneas. Los microbiomas, por ejemplo, están compuestos por diferentes especies microbianas, y el transcriptoma de las células dentro de un tejido puede ser diferente dependiendo del linaje celular. Por estas razones, existe un creciente interés en el desarrollo de métodos de secuenciación decelulas individuales. Para secuenciar el genoma de una sola célula, primero es necesario aislarla físicamente de las células de su entorno. Una solución para esto es a través de la dispersión de la muestra y la encapsulación de células individuales en pequeñas gotas creadas por microfluidos. El objetivo de este trabajo ha sido desarrollar una plataforma de microfluidos que permita la encapsulación de decenas de células por segundo en pequeñas gotas de agarosa con un diámetro de alrededor de 100~\micrometro. Para este propósito, desarrollamos un chip microfluídico en el que se forman gotitas en la intersección entre una solución acuosa y aceite de encapsulación. La agarosa se introduce en la fase acuosa junto con la suspensión celular. Al enfriarse, esta agarosa captura las células individuales. Diseñamos y fabricamos un sistema para mantener la agarosa caliente antes de alcanzar el chip y una platina de microscopio para sujetar el chip durante su observación. También se implementó un \script\ en lenguaje de programación \texttt{Python} para calcular las densidades óptimas de las células en la muestra que se va a encapsular. Los resultados muestran que nuestro sistema encapsuló efectivamente las células, en las densidades previstas. Por lo tanto, este sistema es adecuado para realizar la encapsulación de suspensiones celulares para la secuenciación de una sola célula.

\end{abstract}

\textbf{Palabras clave:} plataforma microfluídrica -- gotas basados en microfluídos -- encapsulado de célula única -- análisis genómico de célula única


\selectlanguage{english}

\begin{abstract}
Next-generation sequencing (NGS) technologies allow the determination of the complete genome or transcriptome of a biological sample. However, many biological samples are heterogeneous. Microbiomes for example are composed of different microbial species, and the transcriptome of cells within a tissue is different depending on the cell lineage. For these reasons, there is growing interest in the development of single-cell sequencing methods. To sequence the genome of a single cell, it is first necessary to physically isolate from other cells. One solution for this is through the dispersion of the sample and encapsulation of individual cells into of small droplets created by microfluidics. The goal of this work has been to develop a microfluidric platform that allows the encapsulation of tens of cells per second in small droplets of agarose that are around 100~\micrometro\ diameter. For this purpose, we developed a microfluidic chip in which droplets are formed by dripping at the intersection between an aqueous solution and the encapsulating oil. Agarose is introduced in the aqueous phase along with the cell  suspension. Upon cooling, this agarose captures the individual cells. We designed and fabricated a system to maintain the agarose warm before reaching the chip and a microscope stage to hold the chip during its observation. A python script was also implemented to calculate the optimum densities of cells in the sample to be encapsulated. Results show that our system effectively encapsulated cells, at the predicted densities. This system is thus suited to perform the encapsulation of cell suspensions for single-cell sequencing.
\end{abstract}

\textbf{Key words:} microfluidic platform -- microfluidics  droplet  -- single-cell encapsulation -- single-cell genome sequencing methods

% \begin{abstract}
% Para secuenciar el genoma de una sola célula es necesario, en primer lugar, aislarla físicamente en un entorno libre de otras células. Una de las formas en que esta abordando el problema recientemente es a través de la creación de pequeñas gotas mediante de pequeños dispositivos cuyo funcionamiento está basados en las propiedades de los microfluídos.

% El propósito de este trabajo ha sido desarrollar una plataforma microflídrica que permite encapsular del orden de decenas de células por segundo en pequeñas gotas de agarosa que rondan los 100 de diámetro. Para ello se ha se ha utilizado un chip en el que las gotas se forman por goteo en la intersección del flujo de la agrosa con aceite mineral. Ha sido necesario diseñar y fabricar un sistema para mantener la agarosa caliente antes de llegar al chip y una la platina de microscopio para sujetar firmemente el chip durante su observación. También se ha elaborado un pequeño script en python para calcular las densidades óptimas de células en la muestra a encapsular en función del diámetro de la gota y la relación entre encapsulados únicos y múltiples que queremos que se produzca.

% Los primeros resultados analizados muestran que efectivamente hemos conseguido encapsular células, concretamente cianobacterias, y las densidades de células en la muestra que estamos calculado permiten un encapsulado con las características deseadas. Se espera que lo desarrollado aquí serva como punto de partida a futuros proyectos para el estudio del genoma de células individuales.

% \end{abstract}


% \textbf{Palabras clave:} Alto desplazamiento al rojo -- galaxias submilimétricas -- lente gravitatoria fuerte -- factor de Bayes -- \mbox{cross-identificación}


% \selectlanguage{english}

% \begin{abstract}
% To sequence the genome of a single cell, it is first necessary to physically isolate it in a free environment from other cells. One of the ways in which the problem is been approach recently is through the creation of small droplets through small devices whose operation is based on the properties of microfluidics.

% The purpose of this work has been to develop a microflidric platform that allows to encapsulate about of tens of cells per second in small droplets of agarose that are around 100 in diameter. For this, a chip has been used in which the droplets are formed by dripping at the intersection of the flow of the agrosa with mineral oil. It has been necessary to design and manufacture a system to keep the agarose warm before reaching the chip and a microscope stage to firmly hold the chip during its observation. A small python script has also been developed to calculate the optimum densities of cells in the sample to be encapsulated as a function of the diameter of the droplet and the relationship between single and multiple encapsulations that we want to produce.

% The first results analyzed show that we have effectively encapsulated cells, specifically cyanobacterias, and the densities of cells in the sample that we are calculating allow an encapsulation with the desired characteristics.  It is hoped that what has been developed here, will serve as a starting point for future projects for the study of the single-cell genome.

% \end{abstract}


% \textbf{Key words:} high-redshift -- submillimiter galaxies -- strong gravitational lensing -- Bayes factors -- \mbox{cross-identification}
